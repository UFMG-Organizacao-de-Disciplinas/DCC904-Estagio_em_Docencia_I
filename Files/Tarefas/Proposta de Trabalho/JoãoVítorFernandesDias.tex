\documentclass[12pt, a4paper]{article}

% Encoding and language
\usepackage[utf8]{inputenc}
\usepackage[T1]{fontenc}
\usepackage[brazil]{babel}
\usepackage{hyperref}
\usepackage{verbatim} % habilita o ambiente comment
\usepackage{indentfirst} % Fazer recuo do primeiro parágrafo
\usepackage{titling} % Reduzir a distância do Título ao topo
\setlength{\droptitle}{-10em} % ajusta a distância do topo

\title{\textbf{Estágio em Docência \\ Proposta de Trabalho}}
\author{João Vítor Fernandes Dias}
\date{\today}

\begin{document}

\begin{comment}

## Moodle

- Enviar a proposta de trabalho no formato PDF pelo Moodle com o nome `<NomeDoAluno>.pdf`
- Ao enviar a proposta, o aluno deve também encaminhar uma mensagem ao coordenador da disciplina por e-mail, incluindo o professor responsável pela disciplina em cópia, de modo a informá-lo da submissão da proposta. A proposta de trabalho será avaliada pelo coordenador da disciplina, compondo a nota final da disciplina ED I/II.

===

## Diretrizes Gerais:

4. Proposta de trabalho: é definida uma data para que o estagiário encaminhe uma proposta de trabalho ao coordenador da disciplina, em que conste também o tipo de atividade a ser conduzida, em comum acordo com o professor responsável pela turma à qual foi alocado.

---

A proposta de trabalho deverá ser previamente acertada com o professor responsável pela turma à qual o estagiário foi alocado, e submetida no Moodle da disciplina em um arquivo PDF (usar o nome NomeCompletoDoAluno.pdf).

A proposta deve conter os seguintes itens:

1. Identificação: Título, Disciplina, Professor responsável pela turma
2. Sobre a Disciplina: Descrição da forma de condução da disciplina adotada pelo professor responsável. Ementa e programa da disciplina. Bibliografia adotada. Formas de avaliação.
3. Atividades: Indicação das atividades e tarefas que ficarão sob a responsabilidade do estagiário.
4. Cronograma: Cronograma de aulas e avaliações, indicando a participação do estagiário ao longo do tempo.

Ao enviar a proposta, o aluno deve também encaminhar uma mensagem ao coordenador da disciplina por e-mail, incluindo o professor responsável pela disciplina em cópia, de modo a informá-lo da submissão da proposta.

A proposta de trabalho será avaliada pelo coordenador da disciplina, compondo a nota final da disciplina ED I/II.

---

### Critérios de Avaliação

A avaliação será feita considerando o planejamento e as atividades realizadas ao longo do semestre. Os critérios são os seguintes:

- Proposta de trabalho: 50 pontos (coordenador da disciplina ED)
  - Conhecimento da disciplina apoiada
  - Planejamento das atividades
  - Atendimento ao conteúdo previsto
\end{comment}

\maketitle

\section{Sobre a Disciplina} \label{sec:disciplina}

% 1. Identificação: Título, Disciplina, Professor responsável pela turma

\textbf{Disciplina:} Introdução a Banco de Dados

\textbf{Horários:} 3ª e 5ª - 19:00/20:40

\textbf{Professor(a):} Rodrygo Luis Teodoro Santos

% \subsection{Condução pelo professor} \label{subsec:conducao}

% 2. Sobre a Disciplina: Descrição da forma de condução da disciplina adotada pelos professores responsáveis. Ementa e programa da disciplina. Bibliografia adotada. Formas de avaliação.

% <Inserir um texto descrevendo a condução pretendida pelo professor (?)>

\subsection{Ementa e Programa da Disciplina} \label{subsec:ementa_e_programa}

% <Copiar detalhes relevantes presentes na Ementa da disciplina>

\begin{itemize}
    \item \textbf{Curso:} Ciência da Computação e Sistemas de Informação;
    \item \textbf{Código:} DCC011;
    \item \textbf{Classificação:} OP;
    \item \textbf{Créditos:} 04;
    \item \textbf{Carga Horária:} (Teórica, Prática, Total) = (60, 00, 60);
    \item \textbf{Requisitos:} Não.
\end{itemize}

\subsubsection{Objetivos} \label{subsubsec:objetivos}

Apresentar os conceitos e os principais aspectos envolvidos no desenvolvimento de aplicações de banco de dados. Ao final do semestre, o aluno deverá Ter adquirido sólidos conhecimentos de banco de dados, sendo capaz de projetar e implementar aplicações usando essa tecnologia, e também entender o funcionamento dos diversos componentes de um sistema de gerência de banco de dados.

\subsubsection{Ementa} \label{subsubsec:ementa}

% <Inserir ementa: basicamente um conjunto de palavras-chaves de conteúdos que serão aprendidos na disciplinas>

Memória auxiliar, organização física e lógica. Métodos de acesso. Estruturas de arquivos. Manipulação de banco de dados. Linguagens e pacotes.Recuperação de informação.

\subsubsection{Programa} \label{subsubsec:programa}

% <Inserir Programa: A ementa, porém mais descritiva>

\begin{itemize}
  \item \textbf{Introdução:}
        Conceitos básicos: banco de dados, sistema de banco de dados, sistema de gerência de banco de dados; Características da abordagem de banco de dados; Modelos de dados, esquemas e instâncias; Arquitetura de um sistema de banco de dados; Componentes de um sistema de gerência de banco de dados.
\item \textbf{Modelos de Dados e Linguagens:}
        Modelo de entidades e relacionamentos: conceitos básicos (entidades, relacionamento e atributos), restrições de integridade, abstrações de dados (especialização, generalização e agregação); Modelo relacional: conceitos básicos (relações, domínios e atributos), restrições de integridade, álgebra relacional, SQL, visões; Modelo de rede e hierárquico: conceitos básicos.
\item \textbf{Projeto de Banco de Dados:}
        O processo de projeto de banco de dados; Mapeamentos do modelo de entidades e relacionamentos para os modelos relacional e de rede; Normalização.
\item \textbf{Aspectos Operacionais:}
        Processamento de consultas; Recuperação de Falhas; Controle de Concorrência; Segurança e Integridade.
\end{itemize}

\subsection{Bibliografia} \label{subsec:bibliografia}

% Eu gostaria de adicionar isso daqui como se fossem as Referências Bibliográficas, mas ainda não pensei numa forma boa de fazê-lo.

% Eu recomendaria criar um `.bib` com todas as referências, citá-los no texto, copiar do PDF gerado para poder ter a formatação bonitinha, e então colar aqui no itemize.

% Pessoalmente gosto dos links para os materiais, sinto que ajuda a quem for ler a encontrar o que deseja (porém talvez leve para arquivos piratas... não sei quão prudente seria essa abordagem.

\textbf{Bibliografia Básica:}

\begin{itemize}
    \item ELMASRI, R. \& NAVATHE, S. B., \textbf{Fundamentals of Database Systems}, 5th Ed.., Addison Wesley, 2006.
\end{itemize}

\textbf{Bibliografia Complementar:}

\begin{itemize}
    \item BATINI, C., CERI, S. \& NAVATHE, S., \textbf{Conceptual Database Design: An EntityRelationship Approach}, Benjamin/Cummings, Redwood City, California, 1992.
    \item DATE, C. J., \textbf{An Introduction to Database Systems}, 6th ed., Addison Wesley, Reading, Mass., 1995.
    \item KORTH, H. F. \& SILBERCHATZ, A., \textbf{Sistemas de Banco de Dados}, 2ª ed. revisada, Makron Books do Brasil, São Paulo, 1995.
    \item RAMAKRISHNAN, R., \textbf{Database Managemente Systems}, WCB/McGraw-Hill, Boston, Mass., 1998.
    \item SILBERCHATZ, A., KORTH, H. F., SUDARSHAN, S., \textbf{Database Systems Concepts}, 3rded., McGraw-Hill, New York, 1996.
    \item ULMANN, J. D., WIDON, J., \textbf{A First Course in Database Systems}, Prentice Hall Upper Saddle River, NJ, 1997
\end{itemize}

\subsection{Formas de Avaliação} \label{subsec:avaliacoes}

\begin{itemize}
    \item Prova 1 ($P_1$): <20> Pontos;
    \item Prova 2 ($P_2$): <20> Pontos;
    \item Trabalho Prático 1 ($TP_1$): <20> Pontos;
    \item Trabalho Prático 2 ($TP_2$): <20> Pontos;
\end{itemize}

$$Nota\ Final = P_1 + P_2 + TP_1 + TP_2$$

% <Inserir os pesos para as pontuações distribuídas pelo professor ao longo do semestre>

% $$NF = 0.1 \cdot Freq + 0.2 P_1 + 0.2 P_2 + TP_1 + TP_2$$

\section{Atividades} \label{sec:atividades}

% 3. Atividades: Indicação das atividades e tarefas que ficarão sob a responsabilidade do estagiário.

% <Inserir atividades e tarefas que serão realizadas pelo estagiário>

\begin{enumerate}
    \item \textbf{Aplicar provas:} Dias <DD/MM/AAAA> e <DD/MM/AAAA>;
    \item \textbf{Elaborar Exercícios:} Trabalho Prático 1 (TP1) sobre a linguagem SQL com correção automática via Moodle;
    \item \textbf{Ministrar aulas}, sendo o conteúdo ``Álgebra Relacional'' nos dias 23/09/2025 e 25/09/2025;
    \item \textbf{Preparar material didático:} atualização do material para o formato Quarto;
    \item \textbf{Oferecimento de motitorias}, sob demanda dos alunos a fim de sanar as dúvidas.
\end{enumerate}

\section{Cronograma} \label{sec:cronograma}

% 4. Cronograma: Cronograma de aulas e avaliações, indicando a participação do estagiário ao longo do tempo.

% <Inserir o cronograma da disciplina em paralelo com as atribuições do estagiário>

\begin{table}[htbp] \centering \caption{Cronograma de Atividades} \label{tab:atividades}
    \begin{tabular}{|c|l|l|l|} \hline
        \textbf{Data} & \textbf{Aula} & \textbf{Tópico} & \textbf{Estagiário} \\ \hline
        
        23/09/2025 & Aula <XX> & Álgebra Relacional & Ministrar aula \\ \hline
        25/09/2025 & Aula <XX> & Álgebra Relacional & Ministrar aula \\ \hline
        XX/XX/2025 & Prova 1 & - & Aplicação de prova \\ \hline
        XX/XX/2025 & Prova 2 & - & Aplicação de prova \\ \hline
    \end{tabular}
\end{table}

\end{document}
