\documentclass[12pt, a4paper]{article}

% Encoding and language
\usepackage[utf8]{inputenc}
\usepackage[T1]{fontenc}
\usepackage[brazil]{babel}
\usepackage{hyperref}
\usepackage{verbatim} % habilita o ambiente comment
\usepackage{indentfirst} % Fazer recuo do primeiro parágrafo
\usepackage{titling} % Reduzir a distância do Título ao topo
\setlength{\droptitle}{-10em} % ajusta a distância do topo

\title{\textbf{Estágio em Docência \\ Proposta de Trabalho}}
\author{<Nome Completo do Aluno>}
\date{\today}

\begin{document}

\begin{comment}

## Moodle

- Enviar a proposta de trabalho no formato PDF pelo Moodle com o nome `<NomeDoAluno>.pdf`
- Ao enviar a proposta, o aluno deve também encaminhar uma mensagem ao coordenador da disciplina por e-mail, incluindo o professor responsável pela disciplina em cópia, de modo a informá-lo da submissão da proposta. A proposta de trabalho será avaliada pelo coordenador da disciplina, compondo a nota final da disciplina ED I/II.

===

## Diretrizes Gerais:

4. Proposta de trabalho: é definida uma data para que o estagiário encaminhe uma proposta de trabalho ao coordenador da disciplina, em que conste também o tipo de atividade a ser conduzida, em comum acordo com o professor responsável pela turma à qual foi alocado.

---

A proposta de trabalho deverá ser previamente acertada com o professor responsável pela turma à qual o estagiário foi alocado, e submetida no Moodle da disciplina em um arquivo PDF (usar o nome NomeCompletoDoAluno.pdf).

A proposta deve conter os seguintes itens:

1. Identificação: Título, Disciplina, Professor responsável pela turma
2. Sobre a Disciplina: Descrição da forma de condução da disciplina adotada pelo professor responsável. Ementa e programa da disciplina. Bibliografia adotada. Formas de avaliação.
3. Atividades: Indicação das atividades e tarefas que ficarão sob a responsabilidade do estagiário.
4. Cronograma: Cronograma de aulas e avaliações, indicando a participação do estagiário ao longo do tempo.

Ao enviar a proposta, o aluno deve também encaminhar uma mensagem ao coordenador da disciplina por e-mail, incluindo o professor responsável pela disciplina em cópia, de modo a informá-lo da submissão da proposta.

A proposta de trabalho será avaliada pelo coordenador da disciplina, compondo a nota final da disciplina ED I/II.

---

### Critérios de Avaliação

A avaliação será feita considerando o planejamento e as atividades realizadas ao longo do semestre. Os critérios são os seguintes:

- Proposta de trabalho: 50 pontos (coordenador da disciplina ED)
  - Conhecimento da disciplina apoiada
  - Planejamento das atividades
  - Atendimento ao conteúdo previsto
\end{comment}

\maketitle

\section{Sobre a Disciplina} \label{sec:disciplina}

% 1. Identificação: Título, Disciplina, Professor responsável pela turma

\textbf{Disciplina:} <Nome da Disciplina>

\textbf{Horários:} <Dia e Hora das Aulas>

\textbf{Professor(a):} <Nome do Professor>

\subsection{Condução pelo professor} \label{subsec:conducao}

% 2. Sobre a Disciplina: Descrição da forma de condução da disciplina adotada pelos professores responsáveis. Ementa e programa da disciplina. Bibliografia adotada. Formas de avaliação.

<Inserir um texto descrevendo a condução pretendida pelo professor (?)>

\subsection{Ementa e Programa da Disciplina} \label{subsec:ementa_e_programa}

<Copiar detalhes relevantes presentes na Ementa da disciplina>

\begin{itemize}
    \item \textbf{Curso:}
    \item \textbf{Classificação:}
    \item \textbf{Créditos:}
    \item \textbf{Carga Horária:}
    \item \textbf{Requisitos:}
    \item \textbf{Objetivos:}
\end{itemize}

\subsubsection{Ementa} \label{subsubsec:ementa}

<Inserir ementa: basicamente um conjunto de palavras-chaves de conteúdos que serão aprendidos na disciplinas>

\subsubsection{Programa} \label{subsubsec:programa}

<Inserir Programa: A ementa, porém mais descritiva>

\subsection{Bibliografia} \label{subsec:bibliografia}

% Eu gostaria de adicionar isso daqui como se fossem as Referências Bibliográficas, mas ainda não pensei numa forma boa de fazê-lo.

% Eu recomendaria criar um `.bib` com todas as referências, citá-los no texto, copiar do PDF gerado para poder ter a formatação bonitinha, e então colar aqui no itemize.

% Pessoalmente gosto dos links para os materiais, sinto que ajuda a quem for ler a encontrar o que deseja (porém talvez leve para arquivos piratas... não sei quão prudente seria essa abordagem.

Bibliografia Básica:

\begin{itemize}
    \item <Inserir bibliografia básica>
\end{itemize}

Bibliografia Complementar:

\begin{itemize}
    \item <Inserir bibliografia complementar>
\end{itemize}

\subsection{Formas de Avaliação} \label{subsec:avaliacoes}

<Inserir os pesos para as pontuações distribuídas pelo professor ao longo do semestre>

% $$NF = 0.1 \cdot Freq + 0.2 P_1 + 0.2 P_2 + TP_1 + TP_2$$

\section{Atividades} \label{sec:atividades}

% 3. Atividades: Indicação das atividades e tarefas que ficarão sob a responsabilidade do estagiário.

<Inserir atividades e tarefas que serão realizadas pelo estagiário>

\begin{enumerate}
    \item <Conduzir disciplina à distância;>
    \item <Construir disciplinas;>
    \item <Corrigir avaliações;>
    \item <Corrigir exercícios;>
    \item <Ministrar aulas;>
    \item <Preparar material didático.>
\end{enumerate}

\section{Cronograma} \label{sec:cronograma}

% 4. Cronograma: Cronograma de aulas e avaliações, indicando a participação do estagiário ao longo do tempo.

<Inserir o cronograma da disciplina em paralelo com as atribuições do estagiário>

\begin{table}[htbp] \centering \caption{Tabela de Atividades} \label{tab:atividades}
    \begin{tabular}{|c|l|l|} \hline
        \textbf{Data} & \textbf{Atividade/Aula} & \textbf{Participação do Estagiário} \\ \hline
        XX/XX & <Aula 1> & <Descrever atividade> \\ \hline
        XX/XX & <Avaliação 1> & <Descrever participação> \\ \hline
        XX/XX & <Aula 2> & <Descrever atividade> \\ \hline
    \end{tabular}
\end{table}

\end{document}
