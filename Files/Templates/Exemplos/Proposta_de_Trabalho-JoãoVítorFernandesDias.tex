\documentclass[12pt, a4paper]{article}

% Encoding and language
\usepackage[utf8]{inputenc}
\usepackage[T1]{fontenc}
\usepackage[brazil]{babel}
\usepackage{hyperref}
\usepackage{verbatim} % habilita o ambiente comment
\usepackage{indentfirst} % Fazer recuo do primeiro parágrafo
\usepackage{titling} % Reduzir a distância do Título ao topo
\setlength{\droptitle}{-10em} % ajusta a distância do topo

\title{\textbf{Estágio em Docência \\ Proposta de Trabalho}}
\author{João Vítor Fernandes Dias (2024711370)}
\date{\today}

\begin{document}

\begin{comment}

## Moodle

- Enviar a proposta de trabalho no formato PDF pelo Moodle com o nome `<NomeDoAluno>.pdf`
- Ao enviar a proposta, o aluno deve também encaminhar uma mensagem ao coordenador da disciplina por e-mail, incluindo o professor responsável pela disciplina em cópia, de modo a informá-lo da submissão da proposta. A proposta de trabalho será avaliada pelo coordenador da disciplina, compondo a nota final da disciplina ED I/II.

===

## Diretrizes Gerais:

4. Proposta de trabalho: é definida uma data para que o estagiário encaminhe uma proposta de trabalho ao coordenador da disciplina, em que conste também o tipo de atividade a ser conduzida, em comum acordo com o professor responsável pela turma à qual foi alocado.

---

A proposta de trabalho deverá ser previamente acertada com o professor responsável pela turma à qual o estagiário foi alocado, e submetida no Moodle da disciplina em um arquivo PDF (usar o nome NomeCompletoDoAluno.pdf).

A proposta deve conter os seguintes itens:

1. Identificação: Título, Disciplina, Professor responsável pela turma
2. Sobre a Disciplina: Descrição da forma de condução da disciplina adotada pelo professor responsável. Ementa e programa da disciplina. Bibliografia adotada. Formas de avaliação.
3. Atividades: Indicação das atividades e tarefas que ficarão sob a responsabilidade do estagiário.
4. Cronograma: Cronograma de aulas e avaliações, indicando a participação do estagiário ao longo do tempo.

Ao enviar a proposta, o aluno deve também encaminhar uma mensagem ao coordenador da disciplina por e-mail, incluindo o professor responsável pela disciplina em cópia, de modo a informá-lo da submissão da proposta.

A proposta de trabalho será avaliada pelo coordenador da disciplina, compondo a nota final da disciplina ED I/II.

---

### Critérios de Avaliação

A avaliação será feita considerando o planejamento e as atividades realizadas ao longo do semestre. Os critérios são os seguintes:

- Proposta de trabalho: 50 pontos (coordenador da disciplina ED)
  - Conhecimento da disciplina apoiada
  - Planejamento das atividades
  - Atendimento ao conteúdo previsto
\end{comment}

\maketitle

\section{Sobre a Disciplina} \label{sec:disciplina}

% 1. Identificação: Título, Disciplina, Professor responsável pela turma

\textbf{Disciplina:} Introdução a Banco de Dados

\textbf{Código:} DCC011

\textbf{Horários:} 3ª e 5ª - 19:00$\sim$20:40

\textbf{Professor:} Rodrygo Luis Teodoro Santos

\textbf{Curso:} Ciência da Computação e Sistemas de Informação

% \textbf{Classificação:} OP

% \textbf{Créditos:} 04

% \textbf{Carga Horária:} (Teórica, Prática, Total) = (60, 00, 60)

% \textbf{Requisitos:} Não

% \subsection{Condução pelo professor} \label{subsec:conducao}

% 2. Sobre a Disciplina: Descrição da forma de condução da disciplina adotada pelos professores responsáveis. Ementa e programa da disciplina. Bibliografia adotada. Formas de avaliação.

% <Inserir um texto descrevendo a condução pretendida pelo professor (?)>

\begin{comment}
\subsection{Ementa e Programa da Disciplina} \label{subsec:ementa_e_programa}

% <Copiar detalhes relevantes presentes na Ementa da disciplina>

\begin{itemize}
    \item \textbf{Curso:} Ciência da Computação e Sistemas de Informação;
    \item \textbf{Código:} DCC011;
    \item \textbf{Classificação:} OP;
    \item \textbf{Créditos:} 04;
    \item \textbf{Carga Horária:} (Teórica, Prática, Total) = (60, 00, 60);
    \item \textbf{Requisitos:} Não.
\end{itemize}
\end{comment}

\subsection{Objetivos} \label{subsec:objetivos}

Enquanto que o objetivo da disciplina ''Introdução a Banco de Dados`` visa introduzir os fundamentos que permitam ao aluno adquirir o domínio básico da tecnologia de banco de dados, o objetivo deste estágio é contribuir positivamente com o fluxo da disciplina, assistir ao professor nas tarefas acordadas e, consequentemente, aprimorar o processo de aprendizado dos discentes.

% Apresentar os conceitos e os principais aspectos envolvidos no desenvolvimento de aplicações de banco de dados. Ao final do semestre, o aluno deverá ter adquirido sólidos conhecimentos de banco de dados, sendo capaz de projetar e implementar aplicações usando essa tecnologia, e também entender o funcionamento dos diversos componentes de um sistema de gerência de banco de dados.

\subsection{Ementa} \label{subsec:ementa}

% <Inserir ementa: basicamente um conjunto de palavras-chaves de conteúdos que serão aprendidos na disciplinas>

\subsubsection*{Conceitos básicos} \label{subsubsec:conceitos}

\begin{itemize}\setlength{\itemsep}{0pt}
    \item \textbf{Conceitos:} banco de dados, sistema de banco de dados, sistema de gerência de banco de dados;
    \item \textbf{Características} da abordagem de banco de dados;
    \item \textbf{Modelos}, \textbf{esquemas} e \textbf{instâncias};
    \item \textbf{Arquitetura} de um sistema de banco de dados;
    \item \textbf{Componentes} de um sistema de banco de dados.
\end{itemize}

\subsubsection*{Modelos de dados e linguagens} \label{subsubsec:modelos}

\begin{itemize}\setlength{\itemsep}{0pt}
    \item \textbf{Modelo entidade-relacionamento (ER):} conceitos básicos, notação gráfica, restrições, extensões;
    \item \textbf{Modelo relacional:} conceitos básicos, restrições;
    \item \textbf{Álgebra relacional:} conceitos básicos, operações;
    \item \textbf{SQL:} conceitos básicos, operações.
\end{itemize}

\subsubsection*{Projeto de banco de dados} \label{subsubsec:projeto}

\begin{itemize}\setlength{\itemsep}{0pt}
    \item \textbf{Visão geral} do processo de projeto;
    \item \textbf{Projeto lógico:} mapeamento ER/relacional, definição de esquemas em SQL, normalização de esquemas.
\end{itemize}

\subsubsection*{Tópicos avançados} \label{subsubsec:topicos}

\begin{itemize}\setlength{\itemsep}{0pt}
    \item \textbf{Aplicações em ciência de dados:} análise exploratória;
    \item \textbf{Gerência de dados massivos:} modelos NoSQL.
\end{itemize}

\subsection{Bibliografia} \label{subsec:bibliografia}

% Eu gostaria de adicionar isso daqui como se fossem as Referências Bibliográficas, mas ainda não pensei numa forma boa de fazê-lo.

% Eu recomendaria criar um `.bib` com todas as referências, citá-los no texto, copiar do PDF gerado para poder ter a formatação bonitinha, e então colar aqui no itemize.

% Pessoalmente gosto dos links para os materiais, sinto que ajuda a quem for ler a encontrar o que deseja (porém talvez leve para arquivos piratas... não sei quão prudente seria essa abordagem.

\subsubsection*{Bibliografia Básica:} \label{subsubsec:bibliografiabasica}

\begin{itemize}\setlength{\itemsep}{0pt}
    \item \href{https://people.inf.elte.hu/kiss/DB/fundamentals-of-database-systems.pdf}{ELMASRI, R. \& NAVATHE, S. B., \textbf{Fundamentals of Database Systems}, 5th Ed., Addison Wesley, 2006}.
\end{itemize}

\subsubsection*{Bibliografia Complementar:} \label{subsubsec:bibliografiacomplementar}

\begin{itemize}\setlength{\itemsep}{0pt}
    \item \href{https://people.inf.elte.hu/kiss/DB/ullman_the_complete_book.pdf}{Garcia-Molina, H.; Ullman, J. D.; Widom, J. \textbf{Database Systems: The Complete Book}, 2nd Ed. Pearson 2008};
    \item \href{https://github.com/pforpallav/school/blob/master/CPSC404/Ramakrishnan%20-%20Database%20Management%20Systems%203rd%20Edition.pdf}{RAMAKRISHNAN, R.; Gehrke, J. \textbf{Database Managemente Systems}, 3rd Ed. McGraw-Hill, 2003};
    \item \href{https://people.vts.su.ac.rs/~peti/Baze%20podataka/Literatura/Silberschatz-Database%20System%20Concepts%206th%20ed.pdf}{Silberschatz, A.; Korth, H. F.; Sudarshan, S. \textbf{Database Systems Concepts}, 7th Ed. McGraw-Hill, 2019}.
\end{itemize}

\subsection{Formas de Avaliação} \label{subsec:avaliacoes}

$$Nota\ Final = P_1 + P_2 + P_3 + TP_1 + TP_2 + EX_1 + EX_2 + EX_3 + EX_4 + EX_5$$

\begin{itemize}\setlength{\itemsep}{0pt}
    \item 02/09/25, \textbf{02 Pts.}, $EX_1$: Exercícios 1 - Modelo ER;
    \item 11/09/25, \textbf{02 Pts.}, $EX_2$: Exercícios 2 - Modelo Relacional;
    \item 18/09/25, \textbf{25 Pts.}, $P_1$: Prova 1 - Conceitos, Modelagem;
    \item 30/09/25, \textbf{02 Pts.}, $EX_3$: Exercícios 3 - Álgebra Relacional;
    \item 09/10/25$\sim$22/10/25, \textbf{05 Pts.}, $TP_1$: Trabalho Prático 1 - SQL;
    \item 14/10/25, \textbf{02 Pts.}, $EX_4$: Exercícios 4 - SQL;
    \item 23/10/25, \textbf{25 Pts.}, $P_2$: Prova 2 - Álgebra, SQL;
    \item 28/10/25$\sim$24/11/25, \textbf{10 Pts.}, $TP_2$: Trabalho Prático 2 - Desenvolvimento;
    \item 04/11/25, \textbf{02 Pts.}, $EX_5$: Exercícios 5 - Projeto;
    \item 13/11/25, \textbf{25 Pts.}, $P_3$: Prova 3 - Projeto, Tópicos;
\end{itemize}

\begin{comment}
\begin{table}[htbp] \centering \caption{Avaliações} \label{tab:Avaliacoes}
    \begin{tabular}{|c|l|l|c|}
    \hline
        Data                   & Nome                        & Conteúdo               & Pontos \\ \hline
        02/09/25               & $EX_1$: Exercícios 1        & Modelo ER              & 02     \\
        11/09/25               & $EX_2$: Exercícios 2        & Modelo Relacional      & 02     \\
        18/09/25               & $P_1$: Prova 1              & Conceitos, Modelagem   & 25     \\
        30/09/25               & $EX_3$: Exercícios 3        & Álgebra Relacional     & 02     \\
        09/10/25$\sim$22/10/25 & $TP_1$: Trabalho Prático 1  & SQL                    & 05     \\
        14/10/25               & $EX_4$: Exercícios 4        & SQL                    & 02     \\
        23/10/25               & $P_2$: Prova 2              & Álgebra, SQL           & 25     \\
        28/10/25$\sim$24/11/25 & $TP_2$: Trabalho Prático 2  & Desenvolvimento        & 10     \\
        04/11/25               & $EX_5$: Exercícios 5        & Projeto                & 02     \\
        13/11/25               & $P_3$: Prova 3              & Projeto, Tópicos       & 25     \\ \hline
    \end{tabular}
\end{table}
\end{comment}

% <Inserir os pesos para as pontuações distribuídas pelo professor ao longo do semestre>

% $$NF = 0.1 \cdot Freq + 0.2 P_1 + 0.2 P_2 + TP_1 + TP_2$$

\section{Atividades do estagiário} \label{sec:atividades}

% 3. Atividades: Indicação das atividades e tarefas que ficarão sob a responsabilidade do estagiário.

% <Inserir atividades e tarefas que serão realizadas pelo estagiário>

\begin{enumerate}
    \item \textbf{Elaborar Exercícios:} Trabalho Prático 1 ($TP_1$) sobre a linguagem SQL com correção automática via Moodle ($\sim$09/10/2025);
    \item \textbf{Aplicar provas (caso necessário):} Dias 18/09/25, 23/10/25 e 13/11/25;
    \item \textbf{Ministrar aulas}, sendo o conteúdo ``Álgebra Relacional'' nos dias 23/09/2025 e 25/09/2025;
    \item \textbf{Preparar material didático:} atualização parcial do material de aula para o formato Quarto ($\sim$06/11/2025);
    \item \textbf{Oferecimento de monitorias}, sob demanda dos alunos a fim de sanar as dúvidas ($\sim$06/11/2025).
\end{enumerate}


\begin{comment}
\section{Cronograma} \label{sec:cronograma} % 4. Cronograma: Cronograma de aulas e avaliações, indicando a participação do estagiário ao longo do tempo.

% <Inserir o cronograma da disciplina em paralelo com as atribuições do estagiário>

\begin{table}[htbp] \centering \caption{Cronograma de Atividades do estagiário} \label{tab:atividades}
    \begin{tabular}{|r|l|l|l|} \hline
        \textbf{Data} & \textbf{Tópico} & \textbf{Estagiário} \\ \hline
        
        18/09/2025 & $P_1$ & Aplicação de prova \\
        23/09/2025 & Aula 11: Álgebra Relacional & Ministrar aula \\
        25/09/2025 & Aula 12: Álgebra Relacional & Ministrar aula \\
        $\sim$09/10/2025 & $TP_1$ & Elaboração do $TP_1$ \\
        23/10/2025 & $P_2$ & Aplicação de prova \\
        13/11/2025 & $P_3$ & Aplicação de prova \\
        $\sim$06/11/2025 & - & Preparar material didático \\ \hline
    \end{tabular}
\end{table}

\subsection{Cronograma de aulas} \label{subsec:cronograma}

\begin{table}[htbp] \centering \caption{Cronograma de aulas} \label{tab:cronograma}
    \begin{tabular}{|c|c|l|c|}
    \hline
        \textbf{Aula} & \textbf{Data} & \textbf{Conteúdo} & \textbf{Atividades} \\ \hline
        01 & ter. 12/08 & Introdução & ~ \\
        02 & qui. 14/08 & Conceitos Básicos & ~ \\
        -- & ter. 19/08 & DCC Week & ~ \\
        -- & qui. 21/08 & DCC Week & ~ \\
        03 & ter. 26/08 & Modelo ER: Conceitos & ~ \\
        04 & qui. 28/08 & Modelo ER: Extensões & ~ \\
        05 & ter. 02/09 & Modelo ER: Exercícios & $EX_1$ \\
        06 & qui. 04/09 & Modelo Relacional: Conceitos & ~ \\
        07 & ter. 09/09 & Modelo Relacional: Restrições & ~ \\
        08 & qui. 11/09 & Modelo Relacional: Exercícios & $EX_2$ \\
        09 & ter. 16/09 & Revisão & ~ \\
        10 & qui. 18/09 & \textbf{Prova 1:} Conceitos, Modelagem & $P_1$ \\
        11 & ter. 23/09 & Álgebra Relacional: Operações Unárias & ~ \\
        12 & qui. 25/09 & Álgebra Relacional: Operações Binárias & ~ \\
        13 & ter. 30/09 & Álgebra Relacional: Exercícios & $EX_3$ \\
        14 & qui. 02/10 & SQL: DQL & ~ \\
        15 & ter. 07/10 & SQL: DDL + DML & ~ \\
        16 & qui. 09/10 & SQL: Tutorial & > $TP_1$ \\
        17 & ter. 14/10 & SQL: Exercícios & $EX_4$ \\
        18 & qui. 16/10 & Tópicos: NoSQL & ~ \\
        19 & ter. 21/10 & Revisão & ~ \\
        20 & qui. 23/10 & \textbf{Prova 2:} Álgebra, SQL & < $TP_1$; $P_2$ \\
        21 & ter. 28/10 & Projeto: Mapeamento & > $TP_2$ \\
        22 & qui. 30/10 & Projeto: Normalização & ~ \\
        23 & ter. 04/11 & Projeto: Exercícios & $EX_5$ \\
        24 & qui. 06/11 & Tópicos: EDA & ~ \\
        25 & ter. 11/11 & Revisão & ~ \\
        26 & qui. 13/11 & \textbf{Prova 3:} Projeto, Tópicos & $P_3$ \\
        27 & ter. 18/11 & $TP_2$: Desenvolvimento (extraclasse) & ~ \\
        -- & qui. 20/11 & Feriado & ~ \\
        28 & ter. 25/11 & $TP_2$: Apresentação & < $TP_2$ \\
        29 & qui. 27/11 & $TP_2$: Apresentação & ~ \\
        30 & ter. 02/12 & \textbf{Prova Substitutiva} & ~ \\
        -- & qui. 04/12 & \textbf{Exame Especial} & ~ \\ \hline
    \end{tabular}
\end{table}
\end{comment}

\section{Assinaturas} \label{sec:assinaturas} % 8. Assinaturas: Indicação das assinaturas necessárias para a formalização da proposta.

\vspace{1.5cm}

\noindent\rule{8cm}{0.4pt}\\
\textbf{João Vítor Fernandes Dias}\\
\textbf{(Estagiário em docência)}

\vspace{1.5cm}

\noindent\rule{8cm}{0.4pt}\\
\textbf{Rodrygo Luis Teodoro Santos}\\
\textbf{(Supervisor da disciplina)}

\end{document}
